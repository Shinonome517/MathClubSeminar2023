\documentclass{classes/myslide}

\title{情報源符号化定理を示そう}
\author{溝口稜太}
\institute{創域理工学部情報計算科学科4年}
\date{September 24, 2023}

\begin{document}

\section{情報源符号化定理を示そう!!}

\begin{frame}
  \titlepage
\end{frame}

\begin{frame}\frametitle{イントロダクション(圧縮の例)}

  「情報源符号化定理」→"圧縮の限界"を示す定理

  \begin{exampleblock}{圧縮の例}
    \begin{align*}
      &000100000110000 \\
      \xrightarrow[圧縮]{} & 03 \ 11 \ 05 \ 12 \ 04 \\
      \xrightarrow[解凍]{} & 000100000110000 
    \end{align*}
  \end{exampleblock}

  \begin{alertblock}{気づくこと}
    出現頻度の偏りが大きければ、より圧縮ができそう
  \end{alertblock}

\end{frame}

\begin{frame}\frametitle{数学的準備(確率論)}
  aaaa
\end{frame}

\end{document}
