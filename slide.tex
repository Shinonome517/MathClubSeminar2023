\documentclass{classes/myslide}
\begin{document}

\section{情報源符号化定理を示そう!!}

\begin{frame}\frametitle{イントロダクション}

  情報源符号化定理→"圧縮の限界"を示す定理

  \begin{exampleblock}{圧縮の例}
    \begin{align*}
      &000100000110000 \\
      &\xrightarrow[圧縮]{} 03 \ 11 \ 05 \ 12 \ 04 \\
      &\xrightarrow[解凍]{} 000100000110000 
    \end{align*}
  \end{exampleblock}

\end{frame}

\begin{frame}\frametitle{数学的準備(確率論)}
  aaaa
\end{frame}

\end{document}
