\documentclass{classes/myslide}

\title{情報源符号化定理を示そう}
\author{溝口稜太}
\institute{創域理工学部情報計算科学科4年}
\date{September 24, 2023}

\begin{document}

\section{情報源符号化定理を示そう!!}

\begin{frame}
  \titlepage
\end{frame}

\begin{frame}\frametitle{イントロダクション(圧縮の例)}

  「情報源符号化定理」→"圧縮の限界"を示す定理

  \begin{exampleblock}{圧縮の例}
    \begin{align*}
      &000100000110000 \\
      \xrightarrow[圧縮]{} & 03 \ 11 \ 05 \ 12 \ 04 \\
      \xrightarrow[復元]{} & 000100000110000 
    \end{align*}
  \end{exampleblock}

  \begin{alertblock}{気づくこと}
    出現頻度の偏りが大きければ、より圧縮ができそう
  \end{alertblock}

\end{frame}

\begin{frame}\frametitle{数学的準備(圧縮のモデル)}
  「文字の生成」のモデルに必要な数学的定義を行う
  \begin{definition}[]
    $\mathcal{X}$:情報源 \ (有限集合)\\
    $X$:$\mathcal{X}$値離散確率変数 \ (可測写像:$\mathrm{prob. sp.} \rightarrow \mathcal{X}$)\\
    $x$:シンボル, $X$の実現値 \ ($ x \in \mathcal{X}$) \\
    $p(x)$:$X$の確率関数 \ ($p(x) \coloneq P(X = x)$)
  \end{definition}
\end{frame}

\begin{frame}\frametitle{数学的準備(圧縮のモデル)}
  「文字列の生成と圧縮・復元」のモデルに必要な数学的定義を行う
  \begin{definition}[]
    $n, m$:シンボル列長,ビット列長 \ ($n, m \in \mathbb{N}$)\\
    $\mathcal{X}^n$:拡大情報源 \ ($\mathcal{X}$の$n$個の直積)\\
    $(X_1, X_2, ..., X_n)$:$\mathcal{X}^n$値離散確率変数 \ ($\mathcal{X}^n$値確率変数の列)\\
    $(x_1, x_2, ..., x_n )$:シンボル列, $(X_1, X_2, ..., X_n)$の実現値 \ ($ x \in \mathcal{X}$の列)\\
    $\tilde{p}(x_1, x_2, ..., x_n )$:$(X_1, X_2, ..., X_n)$の確率関数 \\($\tilde{p}(x_1, x_2, ..., x_n ) \coloneq P(X_1 = x_1, X_2 = x_2,... X_n = x_n)$)
  \end{definition}
\end{frame}

\begin{frame}\frametitle{数学的準備(圧縮のモデル)}
  続き
  \begin{definition}[]
    $\{0, 1\}^n$:符号源 \\
    $(y_1, y_2,... y_n)$:ビット列 \ ($y \in \{0, 1\}$の列)\\
    $C(x_1, x_2, ..., x_n )$:符号化関数 \ (関数:$\mathcal{X}^n \rightarrow \{0, 1\}^n$)\\
    $l(y_1, y_2, ..., y_m )$:符号語長関数 \ (関数:$\{0, 1\}^n \rightarrow \mathbb{N}$) \\
    $D(y_1, y_2, ..., y_m )$:復号化関数  \ (関数:$\{0, 1\}^n \rightarrow \mathcal{X}^n$) \\
    $R$:圧縮率 \ ($R \coloneq E[\frac{1}{n}l(C((X_1, X_2, ..., X_n)))] \in [0, 1]$)
  \end{definition}
\end{frame}

\begin{frame}\frametitle{数学的準備(圧縮のモデル)}
  "適切な圧縮"を定義する
  \begin{definition}[]
    符号化関数$C$が以下を満たすとき
    \begin{multline*}
      \forall n \in \mathbb{N}, \forall (x_1, x_2, ..., x_n ) \in \mathrm{X}^n, \exists D \\ D(C(x_1, x_2, ..., x_n )) = (x_1, x_2, ..., x_n )
    \end{multline*}
    この$C$による圧縮を可逆な圧縮という
  \end{definition}
  \begin{alertblock}{考えたいこと}
    可逆な圧縮の最大効率
  \end{alertblock}
\end{frame}

\begin{frame}\frametitle{数学的準備(エントロピー)}
  aaaa
\end{frame}

\end{document}
